% Koma-Script Basisklasse
\documentclass[a4paper,12pt,pagesize,headsepline,bibtotoc,titlepage]{scrartcl}

\usepackage[utf8]{inputenc}		% direkte Eingabe von Umlauten & Co. (Vorsicht: Encoding im Editor muss auch UTF-8 sein!)

\usepackage[T1]{fontenc}			% T1-Schriften

\usepackage{mathptmx}			% Times/Mathe \rmdefault
\usepackage[scaled=.90]{helvet}	% Skalierte Helvetica \sfdefault
\usepackage{courier}			% Courier \ttdefault

% Zusatzpakete für mehr mathematische Symbole, Einfügen von Grafiken 
% und bessere Bildunterschriften
\usepackage{amsmath,amsthm,amsfonts,graphicx,caption}

% Wenn man direkt mit dem pdflatex eine PDF-Datei erzeugt, sollten diese beiden Pakete eingebunden werden
\usepackage{hyperref} % Hyperlinks anklickbar
\usepackage{ae,aecompl} % bessere Bildschirmschriftarten usw.
\usepackage{epstopdf} % support eps 

\pagestyle{headings}

% Abstand der Kopfzeile vom Text:
\headsep4mm

\typearea[current]{current}     % Satzspiegel neu berechnen

% andere Bildunterschrift mit Hilfe von caption
\renewcommand{\figurename}{Fig.}
\renewcommand{\captionlabelfont}{\bf}

\newcommand{\quot}[1]{{``#1''}}


\title{
	\includegraphics*[width=0.4\textwidth]{hpi_logo_2017.eps}\\
	\vspace{24pt}
	Genetic Doodle Jump
}
\subtitle{
	Seminar\\
	Machine Intelligence with Deep Learning\\
	Fall Semester 2017/2018
}
\author{
	Marcel Jankrift\\
	Fabian Sommer\\
	Toni Stachewicz\\[12pt]
	Supervisor:\\
	Goncalo Mordido,
	Dr. Haojin Yang\\
	Prof. Dr. Christoph Meinel
}
\date{\today}

\begin{document}
\maketitle
\tableofcontents
\newpage


\section{Introduction}
Lorem ipsum dolor sit amet, consectetur adipiscing elit. Morbi sed nunc leo. Nam ac leo venenatis est commodo vehicula. Nulla justo nisl, venenatis id tincidunt id, porta ut felis. Cras eu justo ac nisi ornare commodo placerat at risus. Suspendisse ut urna tellus. Cras ut erat tempus justo aliquam laoreet. Praesent est neque, interdum quis convallis et, gravida sed arcu. Mauris bibendum, dui at ullamcorper luctus, arcu dolor laoreet nisi, ut facilisis dui enim a nisl. Sed odio risus, pulvinar suscipit feugiat pharetra, varius non est. Morbi pellentesque libero eu odio pulvinar semper. Praesent cursus adipiscing metus nec fermentum. Nullam malesuada euismod mi nec tincidunt. Nulla eget auctor velit. Mauris quam odio, blandit sit amet pharetra non, lobortis sed risus. Aliquam nec orci vel dolor suscipit tristique. Etiam quam eros, commodo at iaculis at, molestie eu metus. Maecenas viverra dui non magna suscipit sodales commodo justo iaculis (siehe Abbildung \ref{abb:test}).

\begin{figure}[hbp]
\begin{center}
\includegraphics*[width=0.75\textwidth]{beispiel.png}\\
\caption{Eine Abbildung, Quelle: \cite{BMXNet17}}
\label{abb:test}
\end{center}
\end{figure}

\section{Related Work}

\section{Doodle Jump}
\subsection{Original game}
The video game Doodle Jump is developed for mobile devices by Lima Sky\footnote{\url{http://www.limasky.com/}}. The player --- the so called \quot{doodler} --- has to jump on platforms in order to get a higher score. There are four different types of platforms:
\begin{enumerate}
    \item Regular ones, which neither move nor break
    \item Horizontally moving platforms
    \item Vertically moving platforms
    \item Vanishable ones, which disappear after jumping on it once
    \item Breaking platforms, where the doodler can't jump on
\end{enumerate}
The platforms are pseudo randomly located on the map. Pseudo randomly located means, that there will always be at least one platform, which is reachable from the previous one and non-breakable (See Fig. \ref{abb:doodlejumpgame}). In other words, there is always a way to reach an infinite score. However, the higher the achieved score, the more difficult the game becomes. The difficulty is reached via higher distances between the platforms and the increased usage of moving and vanishable platforms.

The game also includes bonus features, such as propeller hats, jetpacks, rockets, springs or trampolines. These allow the player to jump higher and faster and therefore skip some platforms. Indeed, there are also \quot{enemies} in the game for example monsters, UFOs or black holes. If the doodler hits a monster or comes close to a black hole or UFO, it dies and the game is over. Monsters and UFOs can also be killed by shooting or jumping on them.

To control the doodler the player of the game has to tilt his mobile devices left and right. If the doodler hits the left or right boundary of the game, it will appear on each other's side. Shooting is performed by tapping on the screen. The downrange depends on the tapping position. Once a player falls down and hits the bottom the game is over.

\begin{figure}[h]
\begin{center}
\includegraphics*[width=0.4\textwidth]{images/game}\\
\caption{Screenshot of original Doodle Jump game}
\label{abb:doodlejumpgame}
\end{center}
\end{figure}

\subsection{Simplified implementation}

\section{User Interface}

\section{Network Layout}

\section{Genetic algorithm}

Whether or not the neural network performs well at playing the game very heavily depends on the specific parameters it uses: Each node possesses a bias and each edge has a weight. Even though the network is rather small, there are too many weights and biases to optimize them manually. 

We therefore employ a genetic algorithm to obtain good parameters for the network. At any time, there is a number of different networks, so-called candidates, all playing the game independent of each other. Once they are all done, their performance is evaluated using a fitness function. The best candidates are then selected and combined in different ways to obtain the next generation of candidates. This combination process is called crossover. Finally, some parameters of some candidates are altered (mutated) to make it possible to generate any network. This new generation is then evaluated again, and the process is iterated.

Doodle Jump is a game where each level is generated randomly. We want to end up with a network that is able to play any level well, regardless of the specific generated layout. The performance of any one network on any specific level may not necessarily be indicative of the networks ability to play the game overall. Therefore, each network plays several levels simultaneously, and each level is taken into account when evaluating the performance.

In order to make the performance of different candidates comparable, all candidates of a generation play the same set of levels. However, for each generation a new set of levels is generated. Otherwise, we might end up with a network that excels in playing the specific set of levels used for evaluation but fails at any other level. Therefore, we can avoid overfitting.

We want to choose the different parts of the genetic algorithm so that the learning process is as quick as possible, i.e. the average score achieved by the candidates after 100 generations should be as high as possible.

\subsection{Fitness function}

The fitness function measures the ability of the candidates to play the game well. Doodle Jump already provides a score which measures the height a player achieves. Since the games aim is to maximize that score, we used the score as initial fitness function.

Beyond that, there are many other fitness function one could think of. For example, at the very beginning of the optimization process, many candidates might do nothing at all, while others move a little bit into the direction of the next platform, but not far enough to actually land on it. All candidates would thus achieve a score of zero, and the primitive fitness function would be unable to discriminate between them. Another fitness function, which rewards candidates that got closer to the platform they failed to jump on might speed up the learning process in the first few generations. It would also be helpful later on, when different candidates might fail at the same platform, thus achieving an equal score. However, such a fitness function would include prior knowledge about how to play the game well. We want to avoid using such fitness functions, since for other problems, prior knowledge might not be available, or might not lead to the optimal solution if available. 

One measurement that does not involve any prior knowledge is time. Between two candidates, both achieving similar score, where one jumps once on every platform and the other has to jump twice on every platform before attempting the next jump, we want to prefer the first candidate since he is climbing twice as fast. Therefore, the required time should negatively affect the fitness of candidates.

We tested multiple fitness functions achieving this in different ways:
\begin{itemize}
\item $fitness = score$
\item $fitness = \frac{score}{time}$
\item $fitness = score - time$
\item $fitness = normalized(score) - normalized(time)$
\item $fitness = score^2 - time$
\item $fitness = normalized(score^2 - time)$
\end{itemize}

For each fitness function, we monitored the learning process for the first 100 generations. Since it is a random process with a high variance, we repeated this TODO times for each function. Figure TODO shows the test results.


\subsection{Selection}
\subsection{Crossover}
\subsection{Mutation}
\subsection{Retaining best candidates}

\section{Interesting findings}
\subsection{Disabling passing through walls}
Very good, simple strategy: Always go right -> local optimum.

When disabled and with (score - time) fitness function: Optimum where players suicide as quickly as possible. 
\subsection{Dealing with moving platforms}
\subsection{Dealing with fake platforms}
\subsection{Dealing with springs}
\section{Conclusion and Future Work}

\newpage
%%%% bibtex file %%%%%%
\bibliographystyle{alphadin}
\bibliography{references} 

\end{document}