% Koma-Script Basisklasse
\documentclass[a4paper,12pt,pagesize,headsepline,bibtotoc,titlepage]{scrartcl}

\usepackage[utf8]{inputenc}		% direkte Eingabe von Umlauten & Co. (Vorsicht: Encoding im Editor muss auch UTF-8 sein!)

\usepackage[T1]{fontenc}			% T1-Schriften

\usepackage{mathptmx}			% Times/Mathe \rmdefault
\usepackage[scaled=.90]{helvet}	% Skalierte Helvetica \sfdefault
\usepackage{courier}			% Courier \ttdefault

% Zusatzpakete für mehr mathematische Symbole, Einfügen von Grafiken 
% und bessere Bildunterschriften
\usepackage{amsmath,amsthm,amsfonts,graphicx,caption}

% Wenn man direkt mit dem pdflatex eine PDF-Datei erzeugt, sollten diese beiden Pakete eingebunden werden
\usepackage{hyperref} % Hyperlinks anklickbar
\usepackage{ae,aecompl} % bessere Bildschirmschriftarten usw.
\usepackage{epstopdf} % support eps 

\pagestyle{headings}

% Abstand der Kopfzeile vom Text:
\headsep4mm

\typearea[current]{current}     % Satzspiegel neu berechnen

% andere Bildunterschrift mit Hilfe von caption
\renewcommand{\figurename}{Fig.}
\renewcommand{\captionlabelfont}{\bf}

\title{
	\includegraphics*[width=0.4\textwidth]{hpi_logo_2017.eps}\\
	\vspace{24pt}
	Genetic Doodle Jump
}
\subtitle{
	Seminar\\
	Machine Intelligence with Deep Learning\\
	Fall Semester 2017/2018
}
\author{
	Marcel Jankrift\\
	Fabian Sommer\\
	Toni Stachewicz\\[12pt]
	Supervisor:\\
	Goncalo Mordido,
	Dr. Haojin Yang\\
	Prof. Dr. Christoph Meinel
}
\date{\today}

\begin{document}
\maketitle
\tableofcontents
\newpage


\section{Introduction}
Lorem ipsum dolor sit amet, consectetur adipiscing elit. Morbi sed nunc leo. Nam ac leo venenatis est commodo vehicula. Nulla justo nisl, venenatis id tincidunt id, porta ut felis. Cras eu justo ac nisi ornare commodo placerat at risus. Suspendisse ut urna tellus. Cras ut erat tempus justo aliquam laoreet. Praesent est neque, interdum quis convallis et, gravida sed arcu. Mauris bibendum, dui at ullamcorper luctus, arcu dolor laoreet nisi, ut facilisis dui enim a nisl. Sed odio risus, pulvinar suscipit feugiat pharetra, varius non est. Morbi pellentesque libero eu odio pulvinar semper. Praesent cursus adipiscing metus nec fermentum. Nullam malesuada euismod mi nec tincidunt. Nulla eget auctor velit. Mauris quam odio, blandit sit amet pharetra non, lobortis sed risus. Aliquam nec orci vel dolor suscipit tristique. Etiam quam eros, commodo at iaculis at, molestie eu metus. Maecenas viverra dui non magna suscipit sodales commodo justo iaculis (siehe Abbildung \ref{abb:test}).

\begin{figure}[hbp]
\begin{center}
\includegraphics*[width=0.75\textwidth]{beispiel.png}\\
\caption{Eine Abbildung, Quelle: \cite{BMXNet17}}
\label{abb:test}
\end{center}
\end{figure}

\section{Related Work}

\section{Doodle Jump}
\subsection{Original game}
\subsection{Simplified implementation}
\section{User Interface}

\section{Network Layout}

\section{Genetic algorithm}
\subsection{Fitness function}
\subsection{Selection}
\subsection{Crossover}
\subsection{Mutation}
\subsection{Retaining best candidates}

\section{Interesting findings}
\subsection{Disabling passing through walls}
Very good, simple strategy: Always go right -> local optimum.

When disabled and with (score - time) fitness function: Optimum where players suicide as quickly as possible. 
\subsection{Dealing with moving platforms}
\subsection{Dealing with fake platforms}
\subsection{Dealing with springs}
\section{Conclusion and Future Work}

\newpage
%%%% bibtex file %%%%%%
\bibliographystyle{alphadin}
\bibliography{references} 

\end{document}